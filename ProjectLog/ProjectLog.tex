%%%%%%%%%%%%%%%%%%%%%%%%%%%%%%%%%%%%%%%%%
% Programming/Coding Assignment
% LaTeX Template
%
% This template has been downloaded from:
% http://www.latextemplates.com
%
% Original author:
% Ted Pavlic (http://www.tedpavlic.com)
%
% Note:
% The \lipsum[#] commands throughout this template generate dummy text
% to fill the template out. These commands should all be removed when 
% writing assignment content.
%
% This template uses a Perl script as an example snippet of code, most other
% languages are also usable. Configure them in the "CODE INCLUSION 
% CONFIGURATION" section.
%
%%%%%%%%%%%%%%%%%%%%%%%%%%%%%%%%%%%%%%%%%

%----------------------------------------------------------------------------------------
%	PACKAGES AND OTHER DOCUMENT CONFIGURATIONS
%----------------------------------------------------------------------------------------

\documentclass{article}

\usepackage{fancyhdr} % Required for custom headers
\usepackage{lastpage} % Required to determine the last page for the footer
\usepackage{extramarks} % Required for headers and footers
\usepackage[usenames,dvipsnames]{color} % Required for custom colors
\usepackage{graphicx} % Required to insert images
\usepackage{listings} % Required for insertion of code
\usepackage{courier} % Required for the courier font
\usepackage{lipsum} % Used for inserting dummy 'Lorem ipsum' text into the template
\usepackage{tikz}
\usepackage{keycommand}
\usetikzlibrary{shapes}

% Margins
\topmargin=-0.45in
\evensidemargin=0in
\oddsidemargin=0in
\textwidth=6.5in
\textheight=9.0in
\headsep=0.25in

\linespread{1.1} % Line spacing

% Set up the header and footer
\pagestyle{fancy}
\lhead{\hmwkAuthorName} % Top left header
\chead{\hmwkClass : \hmwkTitle} % Top center head
\rhead{\hmwkClassInstructor} % Top right header
\lfoot{\lastxmark} % Bottom left footer
\cfoot{} % Bottom center footer
\rfoot{Page\ \thepage\ of\ \protect\pageref{LastPage}} % Bottom right footer
\renewcommand\headrulewidth{0.4pt} % Size of the header rule
\renewcommand\footrulewidth{0.4pt} % Size of the footer rule

\setlength\parindent{0pt} % Removes all indentation from paragraphs

%----------------------------------------------------------------------------------------
%	CODE INCLUSION CONFIGURATION
%----------------------------------------------------------------------------------------

\definecolor{MyDarkGreen}{rgb}{0.0,0.4,0.0} % This is the color used for comments
\lstloadlanguages{Perl} % Load Perl syntax for listings, for a list of other languages supported see: ftp://ftp.tex.ac.uk/tex-archive/macros/latex/contrib/listings/listings.pdf
\lstset{language=Perl, % Use Perl in this example
        frame=single, % Single frame around code
        basicstyle=\small\ttfamily, % Use small true type font
        keywordstyle=[1]\color{Blue}\bf, % Perl functions bold and blue
        keywordstyle=[2]\color{Purple}, % Perl function arguments purple
        keywordstyle=[3]\color{Blue}\underbar, % Custom functions underlined and blue
        identifierstyle=, % Nothing special about identifiers                                         
        commentstyle=\usefont{T1}{pcr}{m}{sl}\color{MyDarkGreen}\small, % Comments small dark green courier font
        stringstyle=\color{Purple}, % Strings are purple
        showstringspaces=false, % Don't put marks in string spaces
        tabsize=5, % 5 spaces per tab
        %
        % Put standard Perl functions not included in the default language here
        morekeywords={rand},
        %
        % Put Perl function parameters here
        morekeywords=[2]{on, off, interp},
        %
        % Put user defined functions here
        morekeywords=[3]{test},
       	%
        morecomment=[l][\color{Blue}]{...}, % Line continuation (...) like blue comment
        numbers=left, % Line numbers on left
        firstnumber=1, % Line numbers start with line 1
        numberstyle=\tiny\color{Blue}, % Line numbers are blue and small
        stepnumber=5 % Line numbers go in steps of 5
}

% Creates a new command to include a perl script, the first parameter is the filename of the script (without .pl), the second parameter is the caption
\newcommand{\perlscript}[2]{
\begin{itemize}
\item[]\lstinputlisting[caption=#2,label=#1]{#1.pl}
\end{itemize}
}

%----------------------------------------------------------------------------------------
%	DOCUMENT STRUCTURE COMMANDS
%	Skip this unless you know what you're doing
%----------------------------------------------------------------------------------------

% Header and footer for when a page split occurs within a problem environment
\newcommand{\enterProblemHeader}[1]{
\nobreak\extramarks{#1}{#1 continued on next page\ldots}\nobreak
\nobreak\extramarks{#1 (continued)}{#1 continued on next page\ldots}\nobreak
}

% Header and footer for when a page split occurs between problem environments
\newcommand{\exitProblemHeader}[1]{
\nobreak\extramarks{#1 (continued)}{#1 continued on next page\ldots}\nobreak
\nobreak\extramarks{#1}{}\nobreak
}

\setcounter{secnumdepth}{0} % Removes default section numbers
\newcounter{homeworkProblemCounter} % Creates a counter to keep track of the number of problems

\newcommand{\homeworkProblemName}{}
\newenvironment{homeworkProblem}[1][Problem \arabic{homeworkProblemCounter}]{ % Makes a new environment called homeworkProblem which takes 1 argument (custom name) but the default is "Problem #"
\stepcounter{homeworkProblemCounter} % Increase counter for number of problems
\renewcommand{\homeworkProblemName}{#1} % Assign \homeworkProblemName the name of the problem
\section{\homeworkProblemName} % Make a section in the document with the custom problem count
\enterProblemHeader{\homeworkProblemName} % Header and footer within the environment
}{
\exitProblemHeader{\homeworkProblemName} % Header and footer after the environment
}

\newcommand{\problemAnswer}[1]{ % Defines the problem answer command with the content as the only argument
\noindent\framebox[\columnwidth][c]{\begin{minipage}{0.98\columnwidth}#1\end{minipage}} % Makes the box around the problem answer and puts the content inside
}

\newcommand{\homeworkSectionName}{}
\newenvironment{homeworkSection}[1]{ % New environment for sections within homework problems, takes 1 argument - the name of the section
\renewcommand{\homeworkSectionName}{#1} % Assign \homeworkSectionName to the name of the section from the environment argument
\subsection{\homeworkSectionName} % Make a subsection with the custom name of the subsection
\enterProblemHeader{\homeworkProblemName\ [\homeworkSectionName]} % Header and footer within the environment
}{
\enterProblemHeader{\homeworkProblemName} % Header and footer after the environment
}

%----------------------------------------------------------------------------------------
%	NAME AND CLASS SECTION
%----------------------------------------------------------------------------------------

\newcommand{\hmwkTitle}{Project Log} % Assignment title
\newcommand{\hmwkDueDate}{29 Maart 2015} % Due date
\newcommand{\hmwkClass}{The Giving Game} % Course/class
\newcommand{\hmwkClassTime}{} % Class/lecture time
\newcommand{\hmwkClassInstructor}{University of Amsterdam} % Teacher/lecturer
\newcommand{\hmwkAuthorName}{Julian Ruger 10352783} % Your name

%----------------------------------------------------------------------------------------
%	TITLE PAGE
%----------------------------------------------------------------------------------------

\title{
\vspace{2in}
\textmd{\textbf{\hmwkClass:\ \hmwkTitle}}\\
\textmd{The Giving Game}\\
\normalsize\vspace{0.1in}\small{\hmwkDueDate}\\
\vspace{0.1in}\large{\textit{\hmwkClassInstructor\ \hmwkClassTime}}
\vspace{3in}
}

\author{\textbf{\hmwkAuthorName}}
\date{} % Insert date here if you want it to appear below your name

%----------------------------------------------------------------------------------------

\begin{document}

\maketitle

%----------------------------------------------------------------------------------------
%	TABLE OF CONTENTS
%----------------------------------------------------------------------------------------

%\setcounter{tocdepth}{1} % Uncomment this line if you don't want subsections listed in the ToC

\newpage
\tableofcontents
\newpage

%----------------------------------------------------------------------------------------
%	PROBLEM 1
%----------------------------------------------------------------------------------------

% To have just one problem per page, simply put a \clearpage after each problem

\section{Week 1}
This week I managed to ...

\subsection{Monday 30th of March 2015}
Today I met with my supervisor. We discussed the project plan I made before the start of the poject. We agreed to meet at least once a week. This meeting will take place each monday at 10.30 a.m. with the exception of next week. Next week the meeting will take place on Tuesday at 11.00 a.m. because of Easter. This week I will be working on designing the simulator and the literature study.

\subsection{Tuesday 31th of March 2015}
I used this day to improve my knowlodge of Python. I searched the internet for Python packages and frameworks I might be able to use. For example I found the Qt framework which I will use for the user interface of the simulator. I read the documentation for this framework and for other packages and I did some tutorials.

\subsection{Wednesday 1st of April 2015}
Today I started designing the simulator. I did most of the designing on paper and I put some thoughts in the design document. Tommorow I will continue with the design of the simulotor and try to finnish the design document.

\subsection{Thursday 2nd of April 2015}
I continued working on the design of the simulator. I have the most important parts of the Back-End figured out. I also created some pseudocode for the Agents and the Goods. Tommorow I will create a more visual design.

\subsection{Sunday 5th of April 2015}
Today I added the visual parts of the design. The design is finnished, any changes or additions I come across during the implementation will be documented in the design document.

\section{Week 2}
\subsection{Tuesday 7th of April 2015}
Today I me with my supervisor. We discussed the design document I created last week. I have to pay attention to the following when creating the simulator:

\begin{itemize}
  \item Comunnity percentage
  \item Perishable and durable goods with a perish factor between [0, 1]
  \item Production time of a perishable product
  \item Multiple goods with the same type, for example 8 perishable goods and 2 durable goods.
\end{itemize}
For the theoretical research I will compare the random rule and the balance rule with the goodwill rule, because these rules are special. During the research I will try to find more rules with the same properties.
I also started working on the Python code for the simulator. I have created a simple simulation for the random rule with durable goods. This piece of code will be used as a foundation for the whole simulation.

\subsection{Wednesday 8th of April 2015}
Today I created a simple simulation for perishable goods. I had some trouble with the implementing the goods. Updating and changing the values of the goods was a bit difficult, because they existed in multiple locations. I decided to assign a few agents as default producers of goods which made creating new goods easy, problem solved. I now have a simple simulation for sustainable and perishable goods, both types can be used during the same simulation. For the perish factor I use a value between 0 and MAX\_INTEGER, this is alot easier and more logical than using a value between [0, 1]. The good perishes after \textit{n} times the good has been given. A new good is produced after \textit{m} transactions.
Friday, Saturday and Sunday I will try to figure out how I should implement the calculation of the community effect and in how the balance should be implemented. I also want to try to create a visualisation of the simulation process. If i can see what is happening it will be much more easy to conclude that the simulation is working correctly.

\subsection{Friday 10th of April 2015}
Today I took my time to look at the Qt framework and VisPy. I managed to create a simple input GUI and i created a testfile for VisPy. I figured out how I can implement the Qt framework with the code from the past few days. Tommorow I will try to visualise the input and the ouput of the simulation. At the end of this week I hope I will have a simple simulation with a GUI for the input and the output. If I manage to do this it will be alot easier to create the rest of the simulator, because I will already have a working foundation/framework which allows me to only having to add code on top of the already existing code.

\subsection{Saturday 11th of April 2015}
I continued working on the GUI.

\subsection{Sunday 12th of April 2015}
Today I finnished the simple input and output for the simulation. The foundation is done.

\subsection{Monday 13th of April 2015}
Today I met with my supervisor. I got some new insights. I have to change a few things in the code, which are explained in the design document. I also got some articles to read from my supervisor. This week I will try to find some more literature on my own if I manage to make progress with the simulator.

\subsection{Tuesday 14th of April 2015}
Today I continued working on the compiler and made the necessary changes to the code. I managed to create a more advanced Input interface where the user can set different values for each good. I also managed to print the transaction realtime. This was a struggle, because threads didn't seem to work correctly. After hours I looked back on a different method which didn't seem to work at first. Luckily this same method did work the second time, I have no idea how though. The code is a mess right now, but frustration has taken a toll on me so I will continue with this tomorrow.

\subsection{Wednesday 15th of April 2015}
Today I created a more advanced output interface where the results can be shown realtime. I also figured out how I can show the giving game progress realtime with vispy. Performance is starting to be an issue. I am considering to implement an interface where the user can choose what and when a graph/data needs to be shown. A new window will open and the user can save the results to a file. 

\section{Week 3}


\section{Week 4}


\section{Week 5}


\section{Week 6}


\section{Week 7}


\section{Week 8}


\section{Week 9}


\section{Week 10}


\section{Week 11}


\section{Week 12}




%----------------------------------------------------------------------------------------

\end{document}