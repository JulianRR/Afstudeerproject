%%%%%%%%%%%%%%%%%%%%%%%%%%%%%%%%%%%%%%%%%
% Programming/Coding Assignment
% LaTeX Template
%
% This template has been downloaded from:
% http://www.latextemplates.com
%
% Original author:
% Ted Pavlic (http://www.tedpavlic.com)
%
% Note:
% The \lipsum[#] commands throughout this template generate dummy text
% to fill the template out. These commands should all be removed when 
% writing assignment content.
%
% This template uses a Perl script as an example snippet of code, most other
% languages are also usable. Configure them in the "CODE INCLUSION 
% CONFIGURATION" section.
%
%%%%%%%%%%%%%%%%%%%%%%%%%%%%%%%%%%%%%%%%%

%----------------------------------------------------------------------------------------
%	PACKAGES AND OTHER DOCUMENT CONFIGURATIONS
%----------------------------------------------------------------------------------------

\documentclass{article}

\newcommand{\tab}[1]{\hspace{.2\textwidth}\rlap{#1}}

\usepackage{fancyhdr} % Required for custom headers
\usepackage{lastpage} % Required to determine the last page for the footer
\usepackage{extramarks} % Required for headers and footers
\usepackage[usenames,dvipsnames]{color} % Required for custom colors
\usepackage{graphicx} % Required to insert images
\usepackage{listings} % Required for insertion of code
\usepackage{courier} % Required for the courier font
\usepackage{lipsum} % Used for inserting dummy 'Lorem ipsum' text into the template
\usepackage{tikz}
\usepackage{keycommand}
\usetikzlibrary{shapes}

% Margins
\topmargin=-0.45in
\evensidemargin=0in
\oddsidemargin=0in
\textwidth=6.5in
\textheight=9.0in
\headsep=0.25in

\linespread{1.1} % Line spacing

% Set up the header and footer
\pagestyle{fancy}
\lhead{\hmwkAuthorName} % Top left header
\chead{\hmwkClass : \hmwkTitle} % Top center head
\rhead{\hmwkClassInstructor} % Top right header
\lfoot{\lastxmark} % Bottom left footer
\cfoot{} % Bottom center footer
\rfoot{Page\ \thepage\ of\ \protect\pageref{LastPage}} % Bottom right footer
\renewcommand\headrulewidth{0.4pt} % Size of the header rule
\renewcommand\footrulewidth{0.4pt} % Size of the footer rule

\setlength\parindent{0pt} % Removes all indentation from paragraphs

%----------------------------------------------------------------------------------------
%	CODE INCLUSION CONFIGURATION
%----------------------------------------------------------------------------------------

\definecolor{MyDarkGreen}{rgb}{0.0,0.4,0.0} % This is the color used for comments
\lstloadlanguages{Perl} % Load Perl syntax for listings, for a list of other languages supported see: ftp://ftp.tex.ac.uk/tex-archive/macros/latex/contrib/listings/listings.pdf
\lstset{language=Perl, % Use Perl in this example
        frame=single, % Single frame around code
        basicstyle=\small\ttfamily, % Use small true type font
        keywordstyle=[1]\color{Blue}\bf, % Perl functions bold and blue
        keywordstyle=[2]\color{Purple}, % Perl function arguments purple
        keywordstyle=[3]\color{Blue}\underbar, % Custom functions underlined and blue
        identifierstyle=, % Nothing special about identifiers                                         
        commentstyle=\usefont{T1}{pcr}{m}{sl}\color{MyDarkGreen}\small, % Comments small dark green courier font
        stringstyle=\color{Purple}, % Strings are purple
        showstringspaces=false, % Don't put marks in string spaces
        tabsize=5, % 5 spaces per tab
        %
        % Put standard Perl functions not included in the default language here
        morekeywords={rand},
        %
        % Put Perl function parameters here
        morekeywords=[2]{on, off, interp},
        %
        % Put user defined functions here
        morekeywords=[3]{test},
       	%
        morecomment=[l][\color{Blue}]{...}, % Line continuation (...) like blue comment
        numbers=left, % Line numbers on left
        firstnumber=1, % Line numbers start with line 1
        numberstyle=\tiny\color{Blue}, % Line numbers are blue and small
        stepnumber=5 % Line numbers go in steps of 5
}

% Creates a new command to include a perl script, the first parameter is the filename of the script (without .pl), the second parameter is the caption
\newcommand{\perlscript}[2]{
\begin{itemize}
\item[]\lstinputlisting[caption=#2,label=#1]{#1.pl}
\end{itemize}
}

%----------------------------------------------------------------------------------------
%	DOCUMENT STRUCTURE COMMANDS
%	Skip this unless you know what you're doing
%----------------------------------------------------------------------------------------

% Header and footer for when a page split occurs within a problem environment
\newcommand{\enterProblemHeader}[1]{
\nobreak\extramarks{#1}{#1 continued on next page\ldots}\nobreak
\nobreak\extramarks{#1 (continued)}{#1 continued on next page\ldots}\nobreak
}

% Header and footer for when a page split occurs between problem environments
\newcommand{\exitProblemHeader}[1]{
\nobreak\extramarks{#1 (continued)}{#1 continued on next page\ldots}\nobreak
\nobreak\extramarks{#1}{}\nobreak
}

\setcounter{secnumdepth}{0} % Removes default section numbers
\newcounter{homeworkProblemCounter} % Creates a counter to keep track of the number of problems

\newcommand{\homeworkProblemName}{}
\newenvironment{homeworkProblem}[1][Problem \arabic{homeworkProblemCounter}]{ % Makes a new environment called homeworkProblem which takes 1 argument (custom name) but the default is "Problem #"
\stepcounter{homeworkProblemCounter} % Increase counter for number of problems
\renewcommand{\homeworkProblemName}{#1} % Assign \homeworkProblemName the name of the problem
\section{\homeworkProblemName} % Make a section in the document with the custom problem count
\enterProblemHeader{\homeworkProblemName} % Header and footer within the environment
}{
\exitProblemHeader{\homeworkProblemName} % Header and footer after the environment
}

\newcommand{\problemAnswer}[1]{ % Defines the problem answer command with the content as the only argument
\noindent\framebox[\columnwidth][c]{\begin{minipage}{0.98\columnwidth}#1\end{minipage}} % Makes the box around the problem answer and puts the content inside
}

\newcommand{\homeworkSectionName}{}
\newenvironment{homeworkSection}[1]{ % New environment for sections within homework problems, takes 1 argument - the name of the section
\renewcommand{\homeworkSectionName}{#1} % Assign \homeworkSectionName to the name of the section from the environment argument
\subsection{\homeworkSectionName} % Make a subsection with the custom name of the subsection
\enterProblemHeader{\homeworkProblemName\ [\homeworkSectionName]} % Header and footer within the environment
}{
\enterProblemHeader{\homeworkProblemName} % Header and footer after the environment
}

%----------------------------------------------------------------------------------------
%	NAME AND CLASS SECTION
%----------------------------------------------------------------------------------------

\newcommand{\hmwkTitle}{Simulation scenarios} % Assignment title
\newcommand{\hmwkDueDate}{11th of April 2015} % Due date
\newcommand{\hmwkClass}{The Giving Game} % Course/class
\newcommand{\hmwkClassTime}{ } % Class/lecture time
\newcommand{\hmwkClassInstructor}{University of Amsterdam} % Teacher/lecturer
\newcommand{\hmwkAuthorName}{Julian Ruger} % Your name

%----------------------------------------------------------------------------------------
%	TITLE PAGE
%----------------------------------------------------------------------------------------

\title{
\vspace{2in}
\textmd{\textbf{\hmwkClass:\ \hmwkTitle}}\\
\textmd{The Giving Game}\\
\normalsize\vspace{0.1in}\small{\hmwkDueDate}\\
\vspace{0.1in}\large{\textit{\hmwkClassInstructor\ \hmwkClassTime}}
\vspace{3in}
}

\author{\textbf{\hmwkAuthorName}}
\date{} % Insert date here if you want it to appear below your name

%----------------------------------------------------------------------------------------

\begin{document}

\maketitle

%----------------------------------------------------------------------------------------
%	TABLE OF CONTENTS
%----------------------------------------------------------------------------------------

%\setcounter{tocdepth}{1} % Uncomment this line if you don't want subsections listed in the ToC

\newpage
\tableofcontents
\newpage

%----------------------------------------------------------------------------------------
%	PROBLEM 1
%----------------------------------------------------------------------------------------

% To have just one problem per page, simply put a \clearpage after each problem

\section{1. Introduction}
All scenarios will start with a subgroup size of 2 of the population. During the simulation this size can be changed. The simulator will be able to tell us the largest subgroup possible. The community percentage (the amount of transactions that take place in a subgroup) will be set at percentage of 95. Eventually during the simulation this percentage should go up to a 100 if we want to have a perfect community without any communication with the outside. This goal is probably with most selection rules hard to reach, thus we start with a percentage of 95 to allow some communication with the outside. If 95 is reached we can conclude that there is a community but that it is an imperfect community. (The terms perfect community and imperfect community will be  explained in the thesis. These terms are also open for change if there is a better suiting term.)

\section{2. Definitions}
\subsection{Parameters}
\begin{description}
  \item[N:] The number of agents used in the simulation
  \item[M:] The number of goods used in the simulation
  \item[Perish period:] The perish period is the amount of transactions it takes before a good perishes. For sustainable goods the perish period is 0, because sustainable goods exist forever. For perishable goods the perish period is greater than 0. For example, when a good has a perish period of 3 then this good can be given away 3 times before it perishes.
  \item[Production delay:] The production delay is the time between the perish of a good and its reproduction. The time until the production is decreased by one after every iteration over all agents who are currently holding a good.
  \item[Nominal value:] The nominal value is an indication of how much a good is worth. The nominal value does not change during the simulation.
  \item[Like factor:] The like factor is a number between 0 and 1 which defines how much agent P likes agent Q. The higer the number the more P likes Q the more likely it is that P will give to Q if the like factor is used in the selection rule.
  \item[Selection rule:] The selection rule is an algorithm that decides/calculates the next agent who should receive a good.

\end{description}


\subsection{Selection rules}
\begin{description}
  \item[Random rule:] The receiving agent is randomly chosen.
  \item[Balance rule:] The receiving agent is chosen on the basis of the balance between the giving agent and the receiving agent. The receiving agent Q is chosen if the balance between the giving agent P and Q is the highest.
  \item[Goodwill rule:] The receiving agent is chosen on the basis of the balance between the giving agent and the receiving agent. The receiving agent Q is chosen if the balance between the giving agent P and Q is the lowest.
\end{description}


\subsection{Simulation types}
\begin{description}
  \item[Parallel:] The transactions are executed at the same time.
  \item[One by one:] The transactions are executed one after another.
\end{description}


\section{3. Norm scenarios}
\subsection{Norm 1}

\subsection{Norm 2}

%----------------------------------------------------------------------------------------
%	Random rule
%----------------------------------------------------------------------------------------
\section{4. Random rule}

\subsection{Scenario 1: RR1}
\subsubsection{Parameters:}
\begin{description}
  \item[N:] 100
  \item[M:] 3
  \item[Perish period:] [0, 0, 0]
  \item[Production delay:] -
  \item[Nominal value:] [1, 2, 3]
  \item[Like factor:] 
\end{description}


\subsection{Scenario 2: RR2}
\subsubsection{Parameters:}
\begin{description}
  \item[N:] 100
  \item[M:] 3
  \item[Perish period:] [1, 2, 3]
  \item[Production delay:] [1,2,3]
  \item[Nominal value:] [1, 2, 3]
  \item[Like factor:] 
\end{description}

The choices for these values of the goods are based on the idea of having a small, medium and large product.

\subsection{Scenario 3: RR3}
\subsubsection{Parameters:}
\begin{description}
  \item[N:] 100
  \item[M:] 4
  \item[Perish period:] [1, 2, 0, 0]
  \item[Production delay:] [1, 2, -, -]
  \item[Nominal value:] [1, 2, 1, 2]
  \item[Like factor:] 
\end{description}

The choices for these values of the goods are based on the idea of having a smaller and a larger product.

\subsection{Hypothesis}
I think that all three scenarios will lead to the same results. I don’t think we will see a community effect and all of the transaction will be equally distributed over all the agents even though a computer can’t be fully random. It will be just like throwing a dice. Throwing a dice thousands of times will lead to every side being up approximately the same amount of times. 

%----------------------------------------------------------------------------------------
%	Balance rule
%----------------------------------------------------------------------------------------
\section{5. Balance rule}

\subsection{Scenario 1: BR1}
\subsubsection{Parameters:}
\begin{description}
  \item[N:] 100
  \item[M:] 3
  \item[Perish period:] [0, 0, 0]
  \item[Production delay:] -
  \item[Nominal value:] [1, 2, 3]
  \item[Like factor:] 
\end{description}

\subsubsection{Hypothesis}
I think that eventually the transactions will only take place between 2 agents, because the balance will be the largest between these two. A community effect will arise with a subgroup of size 2.

\subsection{Scenario 2: BR2}
\subsubsection{Parameters:}
\begin{description}
  \item[N:] 100
  \item[M:] 3
  \item[Perish period:] [1, 2, 3]
  \item[Production delay:] [1 ,2, 3]
  \item[Nominal value:] [1, 2, 3]
  \item[Like factor:] 
\end{description}

The choices for these values of the goods are based on the idea of having a small, medium and large product.

\subsubsection{Hypothesis}
I think that eventually the transactions will only take place between a few agents. A community effect will arise with a subgroup of at least 3 agents, because there are 3 producers. Eventually the subgroup will have a maximum size of 6 assuming the producers will have the highest balance with a non-producing agent.

\subsection{Scenario 3: BR3}
\subsubsection{Parameters:}
\begin{description}
  \item[N:] 100
  \item[M:] 4
  \item[Perish period:] [1, 2, 0, 0]
  \item[Production delay:] [1, 2, -, -]
  \item[Nominal value:] [1, 2, 1, 2]
  \item[Like factor:] 
\end{description}

The choices for these values of the goods are based on the idea of having a smaller and a larger product.

\subsubsection{Hypothesis}
I think that eventually the transactions will only take place between a few agents. I think that for the sustainable goods a similar result will arise as in RR1. For the perishable goods I think that a similar result will arise as in RR2. It will be interesting to see if these two subgroups will merge together or wil opperate separately from each other.

%----------------------------------------------------------------------------------------
%	Goodwill rule
%----------------------------------------------------------------------------------------
\section{6. Goodwill rule}
\subsection{Scenario 1: GR1}
\subsubsection{Parameters:}
\begin{description}
  \item[N:] 100
  \item[M:] 3
  \item[Perish period:] [0, 0, 0]
  \item[Production delay:] -
  \item[Nominal value:] [1, 2, 3]
  \item[Like factor:] 
\end{description}

\subsubsection{Hypothesis}
Lets say 

\subsection{Scenario 2: GR2}
\subsubsection{Parameters:}
\begin{description}
  \item[N:] 100
  \item[M:] 3
  \item[Perish period:] [1, 2, 3]
  \item[Production delay:] [1 ,2, 3]
  \item[Nominal value:] [1, 2, 3]
  \item[Like factor:] 
\end{description}

The choices for these values of the goods are based on the idea of having a small, medium and large product.

\subsubsection{Hypothesis}
I think that eventually the transactions will only take place between a few agents. A community effect will arise with a subgroup of at least 3 agents, because there are 3 producers. Eventually the subgroup will have a maximum size of 6 assuming the producers will have the lowest balance with a non-producing agent.

\subsection{Scenario 3: GR3}
\subsubsection{Parameters:}
\begin{description}
  \item[N:] 100
  \item[M:] 4
  \item[Perish period:] [1, 2, 0, 0]
  \item[Production delay:] [1, 2, -, -]
  \item[Nominal value:] [1, 2, 1, 2]
  \item[Like factor:] 
\end{description}

The choices for these values of the goods are based on the idea of having a smaller and a larger product.

\subsubsection{Hypothesis}
I think that eventually the transactions will only take place between a few agents. I think that for the sustainable goods a similar result will arise as in RR1. For the perishable goods I think that a similar result will arise as in RR2. It will be interesting to see if these two subgroups will merge together or wil opperate separately from each other.


%----------------------------------------------------------------------------------------
%	General overview
%----------------------------------------------------------------------------------------
\newpage
\section{General overview}
The nominal values, perish times and perish delays can be any number. Making these values larger will probably not lead to different results. This can be tested by creating a simulation with larger values, but for now this does not have the priority. It is more important to see the behaviours of the agents with more 'basic' values.
\\
\\
For now I decided to use about 3 goods with 100 agents. This number might be a little too small, because it will take some time for the goods to be traded between all agents. This number can be increased if the simulator takes too long to produce meaningful results. For example if we decide to use 5 goods the nominal values will be [1,2,3,4,5] and the perish time and production time will have the same values. Further research will have more different variables to see how much these variables affect the results. But for now the most basic variables will be used to create a foundation of results and see if we need to expand the scenario’s or change something about the simulation/model.
\\
\\
The three scenarios are the same for every selection rule at this point. I want to establish a baseline first and in later simulations extend the experiments. 
\\
\\
Each selection rule should also have a scenario where there are as many goods as there are agents. This way we can see if any of the selection rules are affected by the number of goods. For every parameter we should experiment with a low value and with an extremely high value to see how these parameters affect the simulation.



%----------------------------------------------------------------------------------------

\end{document}