\documentclass[twoside,openright]{uva-bachelor-thesis}

%\usepackage[dutch]{babel}  % uncomment if you write in dutch
\usepackage{graphicx}
\usepackage{url}
\usepackage{mathtools}
\usepackage{amsmath}


% Title Page
\title{The Giving Game}
\author{Julian Ruger}
\supervisors{Peter Weijland (UvA)}
\signedby{}


\begin{document}
\maketitle

\begin{abstract}
This is the abstract. 
\end{abstract}

\tableofcontents

\chapter{Introduction}
I will explain the purpose of this project.

\chapter{Theoretical background}
I will explain the concept of the giving game.

\section{The rules of the game}
The giving game can be used as a game theory so the rules of the game will be explained here.

\section{Paramaters}
Alot of factors/paramets play a part in the model of the giving game. This section and 'The rules of the game' might form one section together.

\begin{description}
  \item[N:] The number of agents used in the simulation
  \item[M:] The number of goods used in the simulation
  \item[Perish period:] The perish period is the amount of transactions it takes before a good perishes. For sustainable goods the perish period is 0, because sustainable goods exist forever. For perishable goods the perish period is greater than 0. For example, when a good has a perish period of 3 then this good can be given away 3 times before it perishes.
  \item[Production delay:] The production delay is the time between the perish of a good and its reproduction. The time until the production is decreased by one after every iteration over all agents who are currently holding a good.
  \item[Nominal value:] The nominal value is an indication of how much a good is worth. The nominal value does not change during the simulation.
  \item[Like factor:] The like factor is a number between -1 and 0 which defines how much agent P likes agent Q. The higer the number the more P likes Q the more likely it is that P will give to Q if the like factor is used in the selection rule.
  \item[Selection rule:] The selection rule is an algorithm that decides/calculates the next agent who should receive a good.

\end{description}

\chapter{Selection rules}
For every transaction the selection rule decides which agent will receive a good. This decision is based on different parameters of the giving game. These selection rules simulate  multiple real world scenarios for example: choosing the agent based on maximizing the profit (goodwill rule).

\section{Random rule}
The random rule is the most basic rule for the giving game. The agent who should receive the good during the transaction is chosen randomly. The random rule simulates an environment where the agents do not care about the value of the goods and do not care about who will receive the goods. The \textit{like factor} is therefore 0 for every agent pair. This leads to the following yield curve.
\\
\textbf{Yield curve of y = a * x + b where a = 0 and b $>$ 0}
\\
This rule is mostly used to see if the giving game enviroment behaves as it should.

\subsection{Hypothesis}
It is expected that there will be no community effect and that there will not be any visible stabilisation of the transactions even though a machine is pseudorandom. The transactions will eventually be more or less equally distributed over all the agents. It will be just like throwing a dice, the next transactions will not be predictable. 

\section{Balance rule}
A more advanced selection rule is the balance rule. The agent who should receive the good during the transaction is chosen based on the balance between the giving agent P and the receiving agent Q. Agent P chooses agent Q if P has the highest balance with Q. If P has the same highest balance with multiple agents then the receiving agent is chosen randomly between these agents. The balance rule simulates an environment where each agent only gives to the agent from who they have received the most. Agent P tries to maximize the number of goods he receives. The balance in this case can be calculated as follows: 
\\
\\
\textit{Balance of P with Q = Number of goods received from Q – Number of goods given to Q} \\
The following rule applies:\\
\textit{Balance of P with Q = - (balance of Q with P)}\\
This calculations of the balance is different from the calculation of the balance for the yield curve.\\




\subsection{Hypothesis}
The expectations are that a community effect will arise with a subgroup consisting of a few agents. The size of this subgroup is based on the type of goods and the amount of goods. All sustainable goods will eventually be traded only between two agents, because these two agents have the highest balance with each other. Every perishable good has a producer, these producers will also be part of a subgroup. If these producers only give each other their goods then the subgroup size will be as large as the amount of producers. If the producers each give to another agent then multiple subgroups will arise with a size of two. It is expected that the maximum size of a subgroup will be the number of producers plus one non-producer who trades with the producers. It will be interesting to see how these subgroups will behave, if they will merge together or operate individually.

\section{Goodwill rule}

\subsection{Hypothesis}

\chapter{Implementation}

\section{Back-end}

\section{Front-end}

\chapter{Experiments}

\section{Random rule}

\subsection{Results}

\section{Balance rule}

\subsection{Results}

\section{Goodwill rule}

\subsection{results}

\chapter{Conclusions and Discussions}

\chapter{Further research}

\chapter{Appendix A}

\section{Simulator manual}

\section{Code Documentation}



\end{document}